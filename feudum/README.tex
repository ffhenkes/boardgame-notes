% Created 2019-10-04 Fri 17:52
\documentclass[11pt]{article}
\usepackage[utf8]{inputenc}
\usepackage[T1]{fontenc}
\usepackage{fixltx2e}
\usepackage{graphicx}
\usepackage{grffile}
\usepackage{longtable}
\usepackage{wrapfig}
\usepackage{rotating}
\usepackage[normalem]{ulem}
\usepackage{amsmath}
\usepackage{textcomp}
\usepackage{amssymb}
\usepackage{capt-of}
\usepackage{hyperref}
\author{Fabio Favero Henkes}
\date{\today}
\title{}
\hypersetup{
 pdfauthor={Fabio Favero Henkes},
 pdftitle={},
 pdfkeywords={},
 pdfsubject={},
 pdfcreator={Emacs 24.5.1 (Org mode 8.3.4)},
 pdflang={English}}
\begin{document}

\tableofcontents

\section{Objetivo}
\label{sec:orgheadline1}

O objetivo do jogo é tornar-se o mais venerável de todo o reino do rei Daniel.

\section{Mapa}
\label{sec:orgheadline2}

Dividido em 6 partes: mar (pequenas ilhas), grandes ilhas, floresta, deserto, terra desolada e montanhas. As seis localidades iniciais são circuladas em rosa.

Tiles de região: para cada região há um conjunto de tiles que servem para avançar nas épocas do jogo que é dividido em 5 (épocas). Ao atingir a 5a época o jogo termina e a pontuação final ocorre.

Gráficos: existem dois gráficos no tabuleiro representando o avanço das épocas (tiles que precisam ser removidos em cada uma para avançar para a próxima) e o gráfico de evolução de área.
(para cada tipo de localidade qual sua escala de evolução, ex: cidade -> feudum).

Barris de vinho: na cor do jogador onde é possível armazenar enxofre para a fermentação de vinho.

Guildas: 6 guildas circundando o tabuleiro.

\includegraphics[width=.9\linewidth]{./img/guilds.png}

(As guildas relacionam-se umas com as outras conforme acima)

\section{Sacola de recursos (mochila)}
\label{sec:orgheadline3}

Todos os recursos aleatórios do jogo são armazenados.

\section{Área do jogador (tableau)}
\label{sec:orgheadline4}

Cada jogador recebe o mesmo set de componentes com uma pequena diferença de recursos.

\begin{itemize}
\item 11 cartas de ação na cor do jogador
\item 3 peões de 6 lados na cor do jogador representando diferenetes personagens relacionados às guildas
\item 7 marcadores de influência na cor do jogador (mais podem ser conseguidos durante a partida)
\item 4 discos na cor do jogador (dois para uso de guardiões no jogo avançado, que possuem um desenho de ponto de vitória)
\item 7 xelins (dinheiro)
\item 7 recursos de comida
\item 3 recursos diferentes quaisquer a escolha do jogador
\item 1 token de sacola na cor do jogador
\end{itemize}

\section{O jogo}
\label{sec:orgheadline5}

O jogo é dividido em 5 épocas nas quais um número indeterminado de rounds pode ser jogado, as épocas avançam conforme os tiles de região são descobertos e ao atingir a 5 época o jogo acaba e são contados os pontos de vitória.


Um round consiste em 5 passos ou fases:

\begin{itemize}
\item 1) Selecionar 4 das 11 cartas disponíveis e executar suas ações em ordem de turno de jogadores;
\item 2) Alimentar os peões no tabuleiro (comida ou vinho)
\item 3) Rolar o dado de progresso, isso é feito pelo jogador inicial e indica qual tile de região deverá ser removido do jogo.
\item 4) Caso atingida a mudança de época o marcador de época é avançado e a pontuação da época executada.
\end{itemize}
O tabuleiro nesse caso é repopulado conforme descrito na mudança época no manual do jogo. Caso a 5 época tenha sido atingida, após a pontuação de época segue-se para a pontuação de final de jogo.
\begin{itemize}
\item 5) Passar o marcador de primeiro jogador em sentido horário e o dado de progresso, caso a 5 época ainda não tenha sido atingida.
\end{itemize}

\section{Ações do jogador}
\label{sec:orgheadline7}

Escolha 4 dentre as 11 cartas disponíveis para jogar no round.

Após a escolha das cartas o jogador inicial randomicamente escolhido começa.

Alguns símbolos podem aparecer nas cartas

\includegraphics[width=.9\linewidth]{./img/symbol.png}

\begin{itemize}
\item Ação normal: qualquer jogador pode executar a ação normal descrita na seção superior da carta.
\item Ação extra: o jogador pode gastar um recurso de salitre para adicionar uma quinta carta a sua mão nesta rodada.
\item Ação sequencial: o jogador pode gastar um recurso de enxofre para jogar duas cartas em um único turno.
\end{itemize}

Caso o jogador assim escolha, poderá executar apenas uma das ações (extra ou sequencial) por rodada, colocando o recurso correspondente em seu token de sacola na sua área de jogador
para demonstrar aos demais jogadores sua opção.
Após a execução da ação o recurso deve ser descartado no sacola de recursos.

\begin{itemize}
\item Habilidade especial: A posse de determinado recurso, ou de um peão específico permite a utlização da habilidade especial.
\end{itemize}

O jogador DEVE jogar ao menos uma carta por turno mas PODE DESISTIR da ação.

\subsection{As 11 cartas de ação}
\label{sec:orgheadline6}


\textbf{1) Ação migrar:} o jogador pode migrar um de seus peões para dentro ou fora do tabuleiro de jogo. Caso opte por migrar um peão para dentro do tabuleiro, deve pagar uma comida.
O peão deve ser colocado ao lado de outro peão do jogador ou um de seus marcadores de influência, caso não possua nenhum peão no tabuleiro coloque o mesmo próximo a uma das localidades
iniciais. Então escolha a face do peão que representa o personagem desejado (fazendeiro, mercador, alquimista, guerreiro, nobre ou monge).

\textbf{1.1) Habilidade especial "Parente distante":} Caso já possua um peão no tabuleiro do tipo alquimista, o jogador pode escolher colocar o peão em alguma das localidades iniciais do jogo,
sem a necessidade de estar junto a outro de seus peões ou marcadores de influência.

Obs: Ao colocar um peão de respectivo personagem no tabuleiro este irá fornecer ao jogador status na respectiva guilda. Verifique sua influência na guilda neste momento. Cada guilda possui critérios
específicos para fornecer mais status aos jogadores (peões, feudums, postos avançados, etc). O jogador com mais influẽncia na guilda recebe o título de mestre e coloca seu marcador de influência
no espaço de 5vps a ser ganho na virada de época, o segundo jogador recebe o título de oficial e coloca seu marcador no espaço de 3vps.

\textbf{2) Ação de repetir:} esta ação permite que o jogador repita a ação de qualquer carta jogada durante a rodada, desde que a mesma possua o símbolo de repetição no canto superior direito.

\textbf{2.1) Habilidade especial "Deja vu":} caso o jogador devolva um salitre para a sacola de recursos é possível executar a carta que não possuir o símbolo de repetição uma segunda
vez na rodada. Cartas com os símbolos 2x riscados (conquista e defesa) nunca podem ser repetidas.

\textbf{3) Ação mover:} sua permissão de movimento total é igual ao número de peões que possui no tabuleiro, estes movimentos podem ser divididos entre todos os peões. O jogador pode se deslocar
por estradas ou rotas de embarcações especiais.

     Furtando recursos: ao final de sua ação de movimento se um de seus peões estiver ao lado de de localidades com recursos vc pode furtar 1 deles a sua escolha, apenas um de seus peões pode
fazê-lo mesmo que hajam mais peões junto a recursos. Não é possível furtar de um local que vc mesmo domine ou que vc possua um servo.
Caso escolha um enxofre o mesmo pode ir para seus suprimentos (tableu) ou para seu barril de vinho.

     Embarcações: se o jogador tiver uma embarcação poderá se deslocar pelas rotas especiais, sendo rota de submarino (bolhas), rota de navio (ondas) ou rota de voo (passáros). O jogador
pode abandonar a embarcação e continuar o movimento pela estrada, mas a mesma poderá ser usada por qualquer um daqui pra frente. Embarcações podem ser adquiridas na guilda do alquimista.

Balsas: Caso o alquimista não possua mais embarcações para a venda, as balsas podem ser usadas mediante o pagamento de 2 xelins.

     Viagem épica: Caso vc jogue a ação mover por duas vezes com sucesso na rodada (vide carta de repetir) avance na trilha de viagem épica ao encontrar um espaço com ícone de carta
compre 2 cartas de escritura real e fique com 1 devolvendo a outra para o fundo do baralho. A viagem é pontuada ao final do jogo sendo que o primeiro jogador na trilha pontua o maior
valor da seção onde se encontra, caso empate ambos pontuam o maior valor. No final da trilha há o monastério, qualquer marcador que o atingir marcará 17pvs.

\textbf{3.1) Habilidade especial "boa sorte":} Caso o jogador possua um peão monge no tabuleiro este pode adicionar +1 ao seu movimento. Escolha um de seus peões monge para a habilidade.
Qualquer peão independente do personagem pode ser movido.

\textbf{4) Ação de Influência:} o jogador pode adicionar um marcador de influência para cada local que houver um de seus peões. Cada localidade pode conter no máximo tres marcadores
pertencentes a dois jogadores.

Soberanos, súditos e servos: O primeiro jogador a adicionar um marcador de influência no local torna-se o soberano, o segundo e diferente jogador coloca o marcador ao lado sendo
o servo garantindo o direito de guardar o terreno. No futuro qualquer um dos dois jogadores pode adicionar um marcador de influência final, caso já seja vc o soberano adicione este
marcador ao lado tornando-se o súdito, caso vc seja o servo, tome o lugar do soberano rebaixando o mesmo a servo e tornando seu outro marcador súdito.

Obs: algumas localidades podem aumentar status em algumas guildas, verifique os status neste momento.

\textbf{4.1) Habilidade especial "dinheiro é influência":} se o jogador tiver o peão de comerciante em um local governado por um único marcador de oponente (ou seja não há súdito), pague 1 xelim
para mover o soberano da região para a posição de servo e coloque um de seus marcadores de influência como soberano na localidade. Mesmo que o jogador tenha múltiplos mercadores em outras
localidades, poderá executar essa habilidade em apenas uma delas.

\textbf{5) Ação de taxação:} caso o jogador seja soberano em uma cidade, colete 2 xelins, ganhe 1 xelim adicional para cada cidade ou feudum adicional que o jogador controle.

\textbf{5.1) Habilidade especial "Reforços":} caso o jogador possua um peão de guerreiro (cavaleiro) no tabuleiro pode também coletar um marcador de influência do estoque geral ou da
guilda dos guerreiros (caso o estoque esteja esgotado). Mesmo que possua múltiplos cavaleiros apenas um marcador é coletado.

\textbf{6) Ação de exploração:} caso o jogador seja o soberano de um posto de controle compre duas cartas de escritura do rei + 1 carta para cada posto avançado adicional e/ou feudum governado e
mantenha uma delas. Devolva as demais cartas para o fundo do baralho.

\textbf{6.1) Habilidade especial "Pás e espadas":} caso o jogador descarte um ferro para a sacola de recursos, compre mais duas cartas de escritura do rei e mantenha uma colocando a outra na parte de baixo do
baralho. Obs: Esta habilidade é executada depois da ação normal concluída, dando uma carta extra ao jogador.

\begin{itemize}
\item Escrituras do rei: há dois tipos de escrituras do rei que são bônus durante o decorrer do jogo. Podem ser recursos, dinheiro ou ainda cartas de ataque.
\end{itemize}
Mandatos vem com selos vermelhos desenhados na carta e usualmente pedem que se pague algo para receber algo em retorno, estas cartas podem ser jogadas a qualquer momento.
Por exemplo em uma carta poderá constar que se for descartado uma comida o jogador receberá 3 xelins.
Ou no caso de cartas de ataque por exemplo pode constar que se o jogador gastar um ferro ganhará +2 de ataque.
Títulos lhe concedem pontos ao final do jogo por objetivos completados. Selos reais (tokens vermelhos iguais aos que ficam junto a guilda dos nobres) habilitam o objetivo na carta.
Por exemplo uma carta pode lhe dar pontos de vitória por localidades nas quais seja soberano em determinada região.

São permitidos manter até 3 escrituras do rei, caso adquira-se mais, alguma deverá ser devolvida ao final do baralho.

\textbf{7) Ação de colheita:} caso o jogador seja soberano em uma fazenda pegue 5 recursos da sacola de recursos e coloque ao lado de uma de suas fazendas. Apenas uma fazenda é afetada por ação.
Receba um recurso adicional para cada feudum, fazenda ou conta de rosário que tiver.
Obs: O máximo de recursos que uma fazenda comporta são 10. Caso uma fazenda possua 10 recursos só poderá receber novos recursos quando baixar desse número.

Favores: ao invés de colocar a colheita completa em uma de suas fazendas o jogador pode escolher comprar os recursos da sacola e colocar a colheira parcial ali e comprar novamente colocando
em seu estoque pessoal, conforme a tabela de colheita descrita na carta. Caso opte pelos favores, por isso ser um ato muito profano, perderá as contas de rosário transformando-as em selos reais.
Vire a conta de rosário para o lado inverso vermelho (selo real).

\textbf{7.1) Habilidade especial "Inspecionar a colheita":} caso o jogador tenha um peão fazendeiro em jogo ele poderá comprar os recursos aleatórios normalmente da colheita e escolher visivelmente seus favores dentre os recursos comprados.

\textbf{8) Ação de melhoria:} caso o jogador seja o soberano de determinada localidade pode-se realizar o upgrade dessa localidade pagando o recurso necessário.

Ex:
posto avançado + 1 recurso de madeira realiza o upgrade para uma fazenda.
fazenda + 1 recurso de ferro realiza o upgrade para uma cidade, nesse caso todos os recursos residentes na fazenda são descartados para a sacola de recursos.
cidade + 1 selo real realiza o upgrade para um feudum do tipo de sua escolha. Ao colocar o feudum no tabuleiro verifique o status da guilda correspondente.

Obs (trilha militar): Ao realizar o upgrade para um feudum o jogador tornar-se-á o vassalo do mesmo e deve agora respeito ao rei Daniel que deve ser demonstrado pelo avanço na trilha militar.

Após o upgrade ser realizado, retire o tile de região (avanço de época) correspondente a época atual ou inferior e a região onde ocorrido e pontue de acordo, caso não exista um tile na região correspondente pegue o tile de qualquer região,
porém não há pontuação nesse caso.

Os tiles de região (avanço de época) podem ser usados de duas formas diferentes: como um recurso curinga ou como um terreno que pode ser usado com a habilidade especial conforme descrito em seguida.

\textbf{8.1) Habilidade especial "Guardar terreno":} devolvendo um recurso de madeira para a sacola de recursos o jogador pode colocar um de seus terrenos coletados ao lado de qualquer local onde seja servo e pontuar 2pv
Apenas um terreno é permitido por localidade. \textbf{Importante:} jogue essa ação em detrimento da ação normal!
Coloque 3 recursos no terreno de acordo com o tipo solicitado, a cada época subsequente, depois de pontuar, adicione mais três recursos, cada vez pode-se optar por coletar todos os recursos ou deixar que se acumulem,
ao coletar os recursos deve-se pagar um para o soberano da localidade como tributo. Ao início de cada época pontue 2pv para cada um de seus terrenos vazios e 4pv para cada terreno com recursos.

\textbf{9) Ação de conquista:} esta ação não pode ser repetida e nem pode ser a última ação do round. Caso o jogador tenha um peão na mesma localização do peão de outro jogador ou feudum. O jogador pode atacar com multiplos de
seus peões, porem só pode mirar em um peão adversário por vez.
Peões possuem +1 ataque +2 defesa, monstros possuem +1 ataque +1 defesa, feudums possuem +2 defesa em adição a isso um súdito contará +1 defesa e um servo contará +1 ataque. Ao atacar diretamente um peão valores de ataque e
defesa de súditos e servos não contam, a luta é mano a mano entre peões. O jogador atacante pode adicionar um único recurso de salitre para aumentar o ataque do peão em +1. Caso o peão atacante seja da guilda dos guerreiros
podem ser adicionados múltiplos salitres aumentando o ataque em +1 para cada salitre utilizado.
Ao vencer um peão o atacante retira o peão do adversário do tabuleiro e recebe 2pv. No caso da conquista de um feudum todos os marcadores de influência (soberano e súdito), mas não o peão, são removidos e o feudum é rebaixado
até o nível de um posto avançado. Caso haja um servo na localidade, este torna-se o soberano no poosto avançado. Cheque status na guilda correspondente neste momento.
Empates vão para a defesa.

Obs (trilha militar): Para cada conquista realizada o jogador pode colocar um token na trilha com o desenho da catapulta e a cada conquista subsequente, caso falhe em uma conquista na época marcada para avaliação na trilha,
 subtraia o número de pvs indicados. Os pvs perdidos são cumulativos caso não haja discos na trilha para suprimi-los.

\textbf{9.1) Habilidade especial "Esfomear o povo":} caso o jogador possua um peão de nobre em uma localidade com um súdito ou servo do oponente, um deles pode ser conquistado e devolvido para o seu dono. O jogador não ganhará pvs
(pois este é um ato desprezível) mas deve adicionar um disco a trilha militar. Após resolver essa ação remova todas as contas de rosário dos frangos na fazenda. Esta ação deve ser executada em detrimento da ação normal.

\textbf{10) Ação de defesa:} essa ação não pode ser repetida, e é a única ação que pode ser jogada fora do seu turno, esta ação dá +1 de defesa sobre qualquer ataque e ainda 1pv. Caso ninguém lhe ataque a carta pode ser jogada em seu
turno para ganho do pv.

\textbf{10.1) Habilidade especial "Imunidade real":} caso o jogador esteja sendo atacado por uma ação "Esfomear o povo" pode lançar mão dessa habilidade para tornar-se imune e ganhar um selo real. Coloque o selo sob o servo ou súdito
que esteja sendo atacado e este estará imune enquanto o selo permanecer com ele. O selo pode ser retirado para ser utilizado em outra ação porém o servo ou súdito perderá assim a sua imunidade.

\textbf{11) Ação da guilda:} permite que o jogador execute umas das três ações disponíveis em qualquer guilda.

Negociar: qualquer jogador pode negociar qual qualquer uma das 6 guildas, mesmo não tendo nenhuma influência. Com exceção da guilda dos fazendeiros os jogadores podem trocar dinheiro por produtos (recursos, embarcações, influência,
selos reais, contas de rosário, etc)

\textbf{Guilda dos mercadores:} é possível comprar até 3 recursos pelo preço disponível no mercado.

\textbf{Guilda dos alquimistas:} embarcações podem ser compradas por 3 xelins (caso não haja embarcações balsas são liberadas para utilização no mapa).

\textbf{Guilda dos guerreiros (cavaleiros):} uma linha completa de marcadores de influência pode ser comprada por 3 xelins.
Caso seja comprado pelo menos 3 marcadores de influência é possível ter controle de um dos monstros, devolva 2 marcadores de influência para o estoque e marque o ícone de monstro selecionado
na guilda com outro marcador restante, este monstro ficará consigo até que seja eliminado em combate.
Há dois tipos de monstros a serpente marinha e o gigante. A serpente marinha é limitada pelas rotas de navio e submarinos. O gigante é limitado pelas rotas de estradas e de máquinas voadoras.
Monstros adicionam +1 ao seu movimento e podem usar armas assim como peões (escrituras do rei).
Peões ficam imóveis até que matem o monstro em sua localidade ou utilize a ação de migrar para remover seu peão do tabuleiro.
Monstros não precisam ser alimentados, não adicionam influência, e não utilizam veículos.
Caso o monstro seja derrotado, devolva o mesmo para a guilda dos guerreiros e retorne seu marcador de influẽncia para seu estoque pessoal.

\textbf{Guilda dos monges:} Uma cota de rosário por 3 xelins.

A negociação sempre é feita da seguinte forma: 1 xelim para o mestre da guilda, 1 xelim para o oficial, 1 xelim para a bolsa do monge (caso compre de alguma guilda do lado oposto)
ou na cesta do fazendeiro (caso compre de alguma guilda do lado oposto).
O procedimento é repetido até o preço ser finalizado. Ex: Pagando 4 xelins para comprar um item do mercador, iriam 1 para o mestre, 1 para o oficial, 1 para a bolsa do monge
e mais 1 para o mestre. Caso os membros estejam ausentes, o dinheiro vai para o estoque exceto a participação do monge e/ou fazendeiro.

Para a guilda dos fazendeiros, caso o jogador seja soberano em uma fazenda ele pode esvaziar a mesma, colocar estes recursos na guildae receber uma comida ou um xelim para cada recurso entregue. O máximo de recursos comportado
pela guilda dos fazendeiros é de 10, caos esse número extrapole e hajam contas de rosário sobre os frangos, 4 desses recursos são então enviados para a guilda dos mercadores, caso falte espaço os restantes vão para a guilda
do alquimista, todos os restantes são devolvidos para a sacola de recursos.

Empurrar: o mestre da guilda pode empurrar recursos de sua guilda para a guilda de sua direita, caso complete 1,2 ou 3 dos objetivos o jogador recebe respectivamente 4, 5 ou 6 pvs.

Ex guilda dos fazendeiros: o objetivo é que  mestre preencha totalmente quaisquer das prateleiras na vertical ou horizontalmente (linhas ou colunas) do mercado na guilda dos mercadores,
ele deve empurrar um máximo de 5 recursos para tentar atingir esses objetivos. Digamos que seja possível completar duas prateleiras, isso daria 5vps para o jogador.

Puxar: o oficial da guilda pode puxar da guilda em sua esquerda recursos para sua própria guilda, da mesma forma que a ação de empurrar, recebendo 3 pvs pelo cumprimento de um único objetivo
e recebendo 2 cartas de escrituras do rei, mantendo uma em sua mão e descartando a outra no final do baralho.

Em adição a isso, toda vez que qualquer dessas ações for realizada (empurrar, puxar) é possível que o jogador adicione um guardião na guilda (disco na cor do jogador com símbolo de ponto de vitória)
em ações futuras o jogador que tiver o guardião receberá 1pv extra ao executar a mesma.

\textbf{11.1) Habilidade especial "Banquetes":} caso algum mestre de guilda ou oficial compartilha com o jogador uma localização e o jogador acha que este mestre ou oficial tem falhado em realizar
seu trabalho (ações de empurrar/puxar) o jogador pode então organizar um banquete para o dito (mestre/oficial). Gaste um enxofre de seu barril de vinho e tome seu lugar como se fosse o dito
(mestre/oficial) na guilda e execute a respectiva ação correspondente ao dito (mestre/oficial).
\section{Alimentar os peões}
\label{sec:orgheadline8}

Gaste uma comida para cada peão no tabuleiro para alimentá-lo por essa rodada, ou um vinho (enxofre) de seu barril para mantê-lo sem fome por duas rodadas, caso opte por isso a defesa e
ataque de seu peão será 0 até que passe o efeito da bebedeira, mantenha o cubo sobre seu peão.

\section{Dado de progresso}
\label{sec:orgheadline9}

O jogador com o token de primeiro jogador rola o dado de progresso e remove o tile de região correspondente.

\section{Avanço de época}
\label{sec:orgheadline10}

Caso atingido o critério para avanço de época, mova o marcador de época e execute a respectiva pontuação.

Guildas pontuam 5pvs para o mestre, 3pvs para o oficial
Terrenos com recursos marcam 4pvs, terrenos vazios marcam 2pvs
Onde o jogador for soberano ou estiver como servo cuidando de um terreno pontua de acordo com o número de marcadores que tiver por região em uma escala de 1/3/5/9/11..pvs
Verifique se a época exige que algum jogador esteja na trilha militar por ser vassalo de algum feudum e desconte o número de pvs caso necessário.
Adicione 3 novos recursos aos terrenos.
Adicione recursos aleatórios da sacola de recursos nas regiões de acordo com a época (veja no mapa na trilha de avanço de época).
Caso seja a 3 época, repopule o tabuleiro de acordo com o setup inicial do jogo (recursos no mapa e guildas).

\section{Primeiro jogador}
\label{sec:orgheadline11}

Caso a 5a época não tenha sido atingida, passe o marcador de jogador inicial e o dado de progresso para o próximo jogador em sentido horário, caso tenha sido atingida, siga para
a pontuação de final de jogo.

Adicione os pontos da viagem épica, 1pv para cada posto avançado, fazenda e cidade, 3 pvs para cada feudum, 3pvs para cada 3 localizações iguais (ex: 3 fazendas) com feudums contando como
curingas, 3pvs para cada 3 xelins que o jogador possua e finalmente verifique as escrituras do rei que possuem selos reais e compute os pvs conseguidos por objetivos atingidos.
\end{document}
