% Created 2022-01-01 Sat 18:10
% Intended LaTeX compiler: pdflatex
\documentclass[11pt]{article}
\usepackage[utf8]{inputenc}
\usepackage[T1]{fontenc}
\usepackage{graphicx}
\usepackage{grffile}
\usepackage{longtable}
\usepackage{wrapfig}
\usepackage{rotating}
\usepackage[normalem]{ulem}
\usepackage{amsmath}
\usepackage{textcomp}
\usepackage{amssymb}
\usepackage{capt-of}
\usepackage{hyperref}
\author{Fabio Favero Henkes}
\date{\today}
\title{}
\hypersetup{
 pdfauthor={Fabio Favero Henkes},
 pdftitle={},
 pdfkeywords={},
 pdfsubject={},
 pdfcreator={Emacs 27.1 (Org mode 9.3)},
 pdflang={English}}
\begin{document}

\tableofcontents

\section{First Train to Nuremberg}
\label{sec:org06463dc}

Levar passageiros e encomendas postais usando trilhos construídos pela sua pequena companhia férrea.

Influência com o governo, o negociante de trens usados e com as grandes companhias férreas da região, são fundamentais para atingir seus objetivos.

\subsection{Fase Um: Cubos de Investimento}
\label{sec:orgfff61bb}

Os cubos de sua cor representam uma parte substacial de capital adquirido junto aos negociantes e donos de terra locais.

Cada jogador pega 12 cubos da sua reserva. Caso houverem menos do que 12 cubos, este deve pegar o houver disponível.

Cubos não utilizados no turno anterior, podem ser guardados para utilização no próximo. O número máximo de cubos por turno é de 15.

\subsection{Fase Dois: Pontos de Influência}
\label{sec:orge49dab7}

Existem 4 cores de influência:

Verde e Vermelho: Influência com as grandes companhias férreas da região. Permite a construção de trilhos nas suas mais proeminentes localizações,
além disso ajuda a convencer os conselheiros a compra de seus trilhos depreciados e desvalorizados que não trazem mais receita.

Marrom: Influência com o negociante de trens usados. Essa influência é necessária para obter bons trens para suas entregas.

Branco: Influência com o governo. Com essa influência poderá ignorar donos de terras que se opuserem aos seus planos de expansão.

Randomicamente adicione um disco de influência em cada box de leilão. O token adicionado aumenta o nível de influência que pode ser conseguido daquela respectiva cor
no leilão subsequente.

\subsection{Fase Três: Leilão por Pontos de Influência}
\label{sec:org2b6c98e}

Existem duas linhas de box de influências para o leilão. Na linha superior o mínimo de 3 cubos de investimento devem ser usados no lance. Na linha inferior 2 cubos.

Ao selecionar um box e adicionar cubos de investimento esta quantidade deve ser superior a quantidade anterior adicionada pelo oponente. Ao ser superado dessa forma
o jogador recolhe os cubos para si.

Se um jogador já tem dois lances em jogo, ele passa a vez. Quando todos os jogadores tiverem dois lances em jogo o leilão acaba com cada um arrematando as respectivas influências
descritas no box de influência.

Após obter as influências registre as mesmas nas respectivas trilhas de influência.

\subsection{Fase Quatro: Ordem de Jogo, Construir, Comprar Trens, Transporte}
\label{sec:org65509ae}

De acordo com a influência conquistada por cada jogador, ajuste o jogador inicial das fases subsequentes.

Para a construção de trilhos, a influência com o governo, a de cor branca, é considerada para a ordem de jogo, quem tem mais vai na frente e assim em diante.

Para comprar trens e mover passageiros e transporte de carga é considerada a influência na cor marrom. Da mesma forma que na ação anterior, quem tem mais vai na frente e assim em diante.

Caso dois ou mais jogadores estejam empatados, o jogador que está no topo da pilha de discos vai na frente.

\subsection{Fase Cinco: Construir Trilhos}
\label{sec:org08f46a9}

Nesta fase cada jogador tem a oportunidade de construir uma série de trilhos formando links. Uma série significa uma sequencia que tem uma origem clara e que não contém bifurcações (todavia bifurcações ainda
podem existir pois os links podem ter sido formados a partir de uma junção com outros links previamente construídos).

Ao construir um trilho, este deve ser posicionado de forma que cruze a fronteira entre duas areas. Podem ser construídos trilhos na quantidade de 0 a 15, conforme o jogador deseje.

Após todos os jogadores terem construídos suas séries de trilhos a fase acaba.

No primeiro turno os trilhos devem partir de uma das cidades contendo uma das estações da companhia verde ou vermelha.

Nos turnos subsequentes, o link pode ser extendido a partir do seu trilho previamente construído.

O jogador pode criar uma bifurcação a partir de um trilho previamente construído, porém não pode criar essas bifurcações durante seu processo de construção. Após finalizar a série de trilhos
formando um link, deve ser possível demonstrar a integridade do mesmo formando uma linha continua do início ao fim.

Seu último trilho não precisa necessariamente ser conectado a uma cidade/estação.

O jogador não pode construir trilhos em direção a uma montanha ou sobre os campos de lúpulo.

A área denominada "O Sul" (The South) só pode ser usada como ponto de partida.

O Nuremberg Norte e Nuremberg Sul nunca pode ser conectados pode um trilho.

É possível construir sobre rios, caso eles sejam fronteiras comuns.

Ao construir um trilho em uma área verde ou vermelha, é necessário pagar uma influência desse tipo em adição aos custos de construção. Caso construa para dentro de uma essas áreas e para fora da mesma área
custa apenas um ponto de influência, o custo não é pago dobrado.

Assim que os trilhos de determinado jogador é vendido para uma das companhias, todas as estações neutras conectadas a esses trilhos passam a ser dessa companhia. Então é possível para qualquer jogador
construir a partir dessas áreas, pagando a devida influência.

É possível que uma cidade esteja conectada às duas companhias (verde e vermelha) nesse caso para construir a partir ou para esse local, deve-se pagar influência às duas companhias.

Caso existam um ou mais donos de terra (peões brancos) em uma área que o jogador deseja construir, deve-se pagar uma influência com o governo para cada um desses donos de terra e devolvê-los a caixa
do jogo.

Ao usar o mapa de Nuremberg pontos extras podem ser reclamados, ao conectar por uma linha contínua as cidades de Nuremberg e Furth, todos os jogadores contribuindo para esse link ganham um ponto extra para
cada trilho de sua cor na formação.

Após finalizar a construção, calcule o custo total em pontos de construção.

O custo para construir entre duas planícies é de 1 ponto.

O custo para construir entre uma planície e um vale é de 2 pontos (apenas no mapa do verso).

o custo para construir sobre um rio é de 2 pontos.

O custo deve ser pago usando cubos de investimento e pontos de influência.

Cada cubo de investimento vale um ponto de construção, assim como cada ponto de influência vale um ponto de construção.

Para usar pontos de influência como pontos de construção, primeiramente os cubos de investimento devem ter sido utilizados.

\subsection{Fase Seis: Comprar Trens e Mover Passageiros e Encomendas Postais/Cerveja}
\label{sec:org9711993}

\subsubsection{Encomendas:}
\label{sec:org7722203}

\begin{itemize}
\item Postais: cubos amarelos
\item Cerveja: cubos cinza
\end{itemize}

\subsubsection{Passageiros}
\label{sec:org4791857}

Existem dois tipos (peões vermelhos e verdes) aqueles que desejam mover-se até uma cidade vermelha e os que desejam mover-se a uma cidade verde.


\subsubsection{Trens}
\label{sec:orgb7ca519}

Para comprar trens é necessária influência com o negociante de trens (cor marrom).

Os trens da primeira linha custam 3 pontos de influência porém fornecem mais pontos de vitória.

Os trens da segunda linha custam 2 pontos de influência.

Os trens da terceira linha custam 1 ponto de influência.

\subsubsection{Ações}
\label{sec:org2d6fc61}

Durante essa fase, escolha entre comprar trens, transportar ou passar.

Caso escolha passar não poderá mais executar ações nessa fase.

Antes de mover as encomendas e os passageiros é necessário ter adquirido o espaço nos trens e uma locomotiva.

Uma vez que o jogador execute uma ação, isso se repete até que cada jogador tenha executado quantas ações desejar ou seja posssível.

É possível que o jogador compre um trem, leve passageiros e encomendas, depois compre outro trem e leve mais passageiros e encomendas e vá indo assim.

A fase só acaba quando todos os jogadores optarem por passar.

Caso a escolha seja comprar um trem, o custo deverá ser pago em influência de acordo com o valor do trem.

No tabuleiro de trens existem 6 linhas cada qual com seu custo associado.

É possível comprar até 3 trens por fase, porém um de cada vez por ação.

Cada trem pode apenas ser comprado por um único jogador.

Não é possível ter mais de um único trem por categoria.

É possível utilizar cubos de investimento e/ou influência de outro tipo na razão de 3:1 para inteirar o valor dos trens e nunca mais do que o necessário para a compra dos trens.

Ao mover passageiros ou encomendas, retire-os do tabuleiro e aloque em seu trem no respectivo espaço correspondente (carga ou vagão).

Nesse jogo as encomendas e os passageiros não são movidos no tabuleiro, porém eles devem efetivamente poder utilizar as linhas construídas corretamente.

Não é possível mover uma encomenda para uma estação que seja de uma das grandes companhias (vermelha ou verde) e coloca-la em seu trem, se for esse o caso retire os cubos de jogo.

Da mesma forma um passageiro não pode mover-se de uma estação de determinada cor para outra de mesma cor, deve ser removido do jogo se esse for o caso.

\subsection{Fase Sete: Pontos de Vitória e Lucros e Perdas}
\label{sec:org89ab913}


\subsubsection{Pontos de Vitória}
\label{sec:org14928d7}

Some os pontos de vitória conquistados pelo número de passageiros transportados nas devidas classes e encomendas entregues.

Marque estes pontos na trilha.


\subsubsection{Lucros e Perdas}
\label{sec:org42fe745}

Seu lucro é contabilizado pelas encomendas e passageiros transportados neste turno menos as despesas com a manutenção da rede de trilhos.

Cada cerveja movida vale \$2 e encomendas postais e passageiros valem \$1.

Subtraia então \$1 para cada trilho de sua cor no tabuleiro. Quando tiver o número final, ajuste o disco de sua cor no display de "Lucros e Perdas" (Profit and Loss).

Após o cálculo retire os passageiros e encomendas dos vagões e os mantenha na sua área de jogo pois serão novamente contabilizados no final do jogo.

Retire também seus marcadores de trem e devolva ao seu estoque pessoal.


\subsection{Fase Oito: Ordem de Jogo}
\label{sec:orgafc455c}

Ajuste a sua ordem de jogo de acordo com o display de Lucros e Perdas" onde a companhia do jogador que estiver com mais lucro (ou menor perda) ficará na frente.

\subsection{Fase Nove: Aquisições}
\label{sec:org2dafef7}

É difícila para uma pequena companhia manter-se lucrativa. Dessa forma para que sua companhia não vá a falência devido ao custo elevado de manutenção da infraestrutura criada por você,
será necessário fazer com que as grandes companhias assumam o controle da sua malha férrea.

Para tal será necessária influência para convencer as grandes que vale a pena assumir o controle de seus trilhos.

Gaste 1 ponto de influência com a respectiva companhia para que a mesma assuma cada dois de seus trilhos.

Todos os trilhos cedidos devem fazer parte de um mesmo link.

O primeiro trilho do link, deve ser conectado com uma estação da companhia de mesma cor que estará fazendo essa aquisição.

O final do trilho deve terminar em uma cidade.

Você pode escolher não ter nenhum de seus trilhos adquirido.

\subsection{Fase Dez: Fim do Turno}
\label{sec:org9b62d94}

Caso seja o último turno vá para a proxima fase.

Avance o marcador de turno.

Remova os tokens de influencia dos boxes e coloque-os de volta na bag.


\subsection{Fim de Jogo e Vencedor}
\label{sec:org8c55b3a}

Uma vez terminado, os jogadores calculam seus pontos de vitória finais.

2VP para cada set formado por 1 cerveja, 1 encomenda postal e 1 passageiro (vermelho ou verde).

Ajuste esse valor no display de "Lucros e Perdas".

No seu track de PV caso esteja em lucro, adicione esse valor ao total, caso em perda, diminua essa valor do seu total.

Diminua do seu total 1 ponto para cada trilho de sua cor remanescente no tabuleiro.

Esse é o seu score final.

O jogador com mais pontos de vitória é o vencedor. Em caso de empate o jogador que estiver na frente do display de "Lucros e Perdas" é o vencedor.
\end{document}
