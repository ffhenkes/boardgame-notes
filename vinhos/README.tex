% Created 2020-05-31 Sun 00:57
% Intended LaTeX compiler: pdflatex
\documentclass[11pt]{article}
\usepackage[utf8]{inputenc}
\usepackage[T1]{fontenc}
\usepackage{graphicx}
\usepackage{grffile}
\usepackage{longtable}
\usepackage{wrapfig}
\usepackage{rotating}
\usepackage[normalem]{ulem}
\usepackage{amsmath}
\usepackage{textcomp}
\usepackage{amssymb}
\usepackage{capt-of}
\usepackage{hyperref}
\author{Fabio Favero Henkes}
\date{\today}
\title{}
\hypersetup{
 pdfauthor={Fabio Favero Henkes},
 pdftitle={},
 pdfkeywords={},
 pdfsubject={},
 pdfcreator={Emacs 26.1 (Org mode 9.1.9)},
 pdflang={English}}
\begin{document}

\tableofcontents

\section{Vinhos: 2010 Reserve (Resumo das regras com foco em 2 jogadores)}
\label{sec:org33d1d24}

Remova aleatóriamente 2 vinhedos.

Aleatóriamente selecione a ordem inicial de jogo.

Em ordem selecionada cada jogador seleciona um vinhedo da região desejada, paga seu custo, ganha o bônus relacionado da região e produz 1 vinho.
Um cubo de notoriedade é adicionado a região do vinho produzido.

\begin{quote}
Tudo que é adquirido do tabuleiro é pago com bagos (cash). O salário dos enólogos é pago via banco.
\end{quote}

O vinho produzido leva em consideração a qualidade base do mesmo, alguns modificadores podem incrementar esse valor (vinícolas por exemplo).

A ordem de turno é novamente alterada, levando em consideração a região de menor número no mapa, do menor para o maior.

Os tile de previsão do tempo dirão quais condições climáticas serão enfrentadas no ano, quais características são tendência do ano para apreciação na feira e quais as preferências dos magnatas.

No primeiro ano o tile 0 é utilizado onde o clima não terá impacto e os magnatas não se importam com as startups do vinho.

\subsection{Início do ano}
\label{sec:orgf76d2b9}

Descarte o tile de previsão do tempo (obviamente não no ano 1) e coloque o marcador de impostos (marcador do turno) no espaço inicial.

Em ordem de turno cada jogador seleciona uma ação.

Todos iniciam o jogo no espaço central do "Quadrel" e selecionam a ação desejada.

Ações adjacentes (diagonais contam) ao espaço onde se encontra o jogador são gratuitas.
Ações não adjacentes custam 1 bago pago ao banco.
Ações onde o meeple de outro jogador se encontre custam 1 bago, pago ao respectivo jogador.
Ações onde esteja o marcador de impostos (marcador do turno) custam 1 bago, pago ao banco.

Estes efeitos podem ocorrer simultaneamente caso coincidentes.
Ex: Maria usa a ação onde se encontra João e lhe paga um bago, nesse local também existe o marcador de impostos, Maria também paga um bago ao banco, gastando um total de dois bagos.

O jogador nunca pode executar a mesma ação novamente, o movimento para outra ação é sempre obrigatório. A exceção é no centro do "Quadrel" onde é possível permancer sem executar nenhuma ação (plan better).

Quando um jogador realiza a ação de passar/coletiva de imprensa, ele seleciona a sua própria ordem de turno para a ação sub-sequente.

Após a realização da segunda ação no ano o marcador passa para a fase de manutenção.

\subsection{Ações}
\label{sec:org0a0bfe9}

\subsubsection{1) Vinhedos: Compre quantos quiser, mas apenas 1 por região.}
\label{sec:orgcd697e6}

\begin{itemize}
\item Pegue o vinhedo do topo da pilha da região desejada.
\item Pague o custo em bagos (cash).
\item Coloque em seu tabuleiro em um lote vazio de uma de suas fazendas, ou com outro vinhedo da mesma região do mapa e do mesmo tipo de vinho (branco ou tinto).
\end{itemize}

Caso seja o primeiro vinhedo na sua fazenda, coloque seu disco na região do mapa de onde o vinhedo fora comprado. Se possível adicione um cubo de notoriedade à região.

\begin{itemize}
\item Pegue o bônus da região caso se aplique.
\end{itemize}

\subsubsection{2) Vinícolas: Construa 1 ou 2 vinícolas}
\label{sec:orgd84a30c}

Para cada vinícola construída o jogador deve:

\begin{itemize}
\item Pegar 1 tile de vinícola.
\item Pagar 3 bagos (cash).
\item Colocar a vinícola em um lote livre em uma fazenda à sua escolha em seu tabuleiro de jogador.
\end{itemize}

Cada fazenda pode ter no máximo 2 vinícolas.
Não é obrigatório que exista um vinhedo na fazenda.

Se possível adicione um cubo de notoriedade à região selecionda (caso exista um vinhedo na fazenda, dessa região), se não for, adicione o cubo à própria vinícola e aguarde o momento onde um vinhedo seja adicionado à fazenda
para colocar esse cubo na respectiva região.

\subsubsection{3) Enólogos: Contrate 1 ou 2 Enólogos.}
\label{sec:orgb9a1137}

Enólogos trabalham em vinícolas, adicionam 2 à qualidade do vinho produzido. Podem mover-se entre fazendas antes da produção.

Para cada Enólogo contratado:

\begin{itemize}
\item Pague 1 bago (cash).
\item Coloque o mesmo sobre uma vinícola, cada vinícula comporta 1 Enólogo.
\end{itemize}

\begin{quote}
Fazendeiros não podem ser contratados neste modo de jogo (2010 Reserve) a única forma de obte-lôs é na região de \textbf{Ribadejo}.
\end{quote}

\textbf{Para que as pŕoximas ações sejam melhor compreendidas, alguns conceitos precisam ser apresentados:}


\textbf{Envelhecendo e produzindo vinho:}

Ao final de cada ano antes da produção mova dos vinhos existentes um slot para a direita em seu depósito ou adega, representando o envelhecimento.
Quando não houver mais espaço o vinho se transforma em vinagre e é perdido.

Depois disso é possível mover cada Enólogo para uma vinícola livre.

Fazendeiros podem ser movidos para um vinhedo livre (ver a região de Ribadejo para o uso de fazendeiros).

Então a produção de vinho é iniciada:

\begin{itemize}
\item Calcule a qualidade do vinho.
\item Pegue um tile do número apropriado.
\item Coloque esse tile no espaço mais à esquerda de seu depósito ou adega. Com o lado de vinho branco ou tinto, conforme a produção do respectivo vinhedo.
\end{itemize}

\textbf{Qualidade do Vinho}

\begin{itemize}
\item 2 pontos para cada vinhedo na fazenda.
\item 1 ponto para cada fazendeiro na fazenda.
\item 1 ponto para cada vinícola na fazenda.
\item 2 pontos para cada enólogo na fazenda.
\item 3 pontos se estiver fazendo vinho do porto (ver a região de  Douros para o uso do vinho do porto).
\item Condições climáticas de +2 até -2
\end{itemize}

\begin{quote}
Lembrete: Cada vinhedo pode comportar apenas 1 fazendeiro e cada vinícula apenas 1 enólogo.
\end{quote}

\textbf{Valor do vinho}

\begin{itemize}
\item Qualidade do vinho +
\item 1,3,5 de envelhecimento na adega +
\item Opcionalmente 1 ou 2 pontos por cubo de notoriedade descartado da respectiva região do vinho produzido +
\item 1 ponto caso o vinho seja da região de \textbf{Algarve}
\end{itemize}

\subsubsection{4) Adegas: Construa 1 adega}
\label{sec:org2b7a757}

Ajudam que o vinho envelheça mais, aumentando seu valor. Quanto mais velho o vinho, mais valioso se torna.

\begin{itemize}
\item Pegue o tile de adega.
\item Pague 2 bagos (cash).
\item Cubra um de seus depósitos e mova os vinhos existentes para a adega.
\item Se possível coloque um cubo de notoriedade em um espaço disponível na respectiva região.
\end{itemize}

Caso não exista um vinhedo na fazenda onde a adega foi colocada, deixe o cubo de notoriedade sobre a mesma, até que um vinhedo seja adquirido e colocado nessa fazenda.

\subsubsection{5) Vendas (mercado interno): Venda quantos vinhos quiser.}
\label{sec:orgfd29478}

Para cada vinho vendido:

\begin{itemize}
\item Coloque 1 barril em um slot vazio de sua escolha.
\end{itemize}

O vinho deve ser do mesmo tipo (branco ou tinto) e ter no mínimo o mesmo valor do referido slot.
Uma vez vendido o barril não poderá ser movido.

\begin{itemize}
\item Opcionalmente remova 1 ou 2 cubos de notoriedade da região de produção de seu vinho.
\end{itemize}

Apenas faça isso para atingir o valor necessário para a venda, não é permitido usar o cubo sem efeito prático apenas para restringir o uso pelo oponente.

\begin{itemize}
\item Descarte o tile do vinho.
\item Receba o valor da venda como crédito na conta do banco.
\end{itemize}

\textbf{Recupere pares de barris dos estabelecimentos locais:}

Durante seu turno, antes de realizar sua ação, é possível recuperar \textbf{pares} de barris dos estabelecimentos locais, cada par deve vir do mesmo estabelecimento.

Para cada par recuperado coloque 2 cubos de notoriedade em regiões diferentes do mapa a sua escolha, 1 em cada.

\subsubsection{6) Exportação: Exporte quantos vinhos quiser}
\label{sec:org581dcc7}

A exportação provê pontos imediatos e pontuação de final de jogo.

\begin{quote}
Onde houver pontos circulados com uma arte em verde, estes são ganhos imediatamente. Onde essa arte for roxa, são pontuação para o final do jogo.
\end{quote}

\begin{itemize}
\item Coloque 1 barril em um slot vazio do mercado de exportação. Este deve ser no mínimo o valor indicado no slot, cubos de notoriedade são permitidos para aumentar o valor do vinho, as regras seguem confirme descrito acima em vendas.
\item Decarte o tile de vinho.
\end{itemize}

\begin{quote}
No jogo com 2 jogadores apenas os slots da área demarcada interna são válidos.
\end{quote}

\subsubsection{7) Banco}
\label{sec:org1247335}

\begin{itemize}
\item Retirar seus bagos do banco: Mova seu marcador para trás o número de espaços relativo ao valor do saque e pegue este montante em bagos.
\end{itemize}

Caso atinja o valor de -2PV imediatamente perca esses pontos. Não será mais possível realizar saques até que o saldo seja novamente positivo.

\begin{itemize}
\item Depositar seus bagos no banco: Mova seu marcador para a frente o número de espaços relativo ao montante que deseja depositar e entregue seus bagos ao banco.
\end{itemize}

O valor máximo que o banco suporta são 24 bagos na conta do jogador.

\begin{itemize}
\item Investimento: Invista seus bagos no banco, o valor é indicado no tabuleiro. Na fase de manutenção receba juros pelo seu investimento como crédito no banco.
\end{itemize}

\subsubsection{8) Especialistas (Sommelier)}
\label{sec:org68a2704}

Cada especialista pode ajudá-lo em determinada característica na feira do vinho.

Além disto cada um possui uma habilidade bônus que pode ser usada imediatamente. Para isso vire o tile do especialista para oseu verso e faça uso do bônus.

Após passada a próxima feira vire o especialista de volta para o lado da frente, a habilidade pode ser usada novamente, ou o especialista poderá participar da feira seguinte.

\subsubsection{9) Passar/Coletiva de imprensa}
\label{sec:orgb7bcd0b}

\begin{itemize}
\item Passe e não faça nada
\end{itemize}

ou

\begin{itemize}
\item Dê uma coletiva de imprensa.
\end{itemize}

Introduza seu vinho na próxima feira de vinhos.
Esta ação só pode ser realizada uma única vez por feira. Caso não faça a introdução de seu vinho por meio da coletiva, deverá fazê-lo diretamente no momento da feira.

\begin{itemize}
\item Declare qual vinho será apresentado na feira.
\item Opcionalmente remova 1 ou 2 cubos de notoriedade da região de onde o vinho é proveniente para aumentar seu valor.
\end{itemize}

Isto pode ser necessário para atingir a expectativa do magnata Bruno ou apenas para apresentar um vinho e maior valor na feira.

\begin{itemize}
\item Ganhe os pontos de feira caso seu vinho seja de valor superior a 9, ganhe pontos pelo excedente.
\item Pegue seu tile de feira de acordo com a tabela em seu tabuleiro pessoal de jogador.
\item Escolha um quiosque de feira vazio e coloque seu tile nele. Pegue o bônus ofertado nele e ganhe pontos de feira de acordo com as características do vinho marcadas no quiosque. Ex: Aroma (verde), aparência (azul).
\end{itemize}

\begin{quote}
Características do vinho são trilhas na área de feira que representam os atributos de aroma, aparência, percentual de álcool e sabor e são incrementadas de acordo com o que consta no tile de previsão do tempo e com o
uso de especialistas durante a feira.
\end{quote}

\begin{itemize}
\item Coloque barris opcionalmente nos magnatas que consiga cumprir as exigências de apreciação do vinho. No máximo 2 dos três magnatas pode ser atendido.
\end{itemize}

Anabela: seu interesse é o tipo do vinho (branco ou tinto).
Bruno: seu interesse é o valor do vinho.
Carolina: seu interesse é pela região de onde o vinho é proveniente.

Tendo barris em um magnata lhe permite fazer ações bônus desse magnata ao custo de um vinho qualquer.

\subsection{Manutenção}
\label{sec:orgdd824b7}

\begin{itemize}
\item Receba ou pague juros de acordo com seu investimento no banco (movimentação da conta).
\end{itemize}

Se não puder pagar, pague o que for possível com seu crédito no banco até atingir os -2PV, perca esses pontos, descarte uma vinícola, se não tiver como, descarte um vinhedo.
Enólogos e fazendeiros movem-se opcionalmente para outras vinículas ou vinhedos em outras de suas fazendas ou caso não tenham alojamento, são descartados para o estoque geral.

\begin{itemize}
\item Pague o salário dos Enólogos. Demita todos os que não possa ou não queira pagar.
\end{itemize}

\begin{quote}
Todos estas movimentações são feitas no banco. Bagos não são utilizados nesses casos.
\end{quote}

\subsection{Produção}
\label{sec:org57bf658}

\begin{itemize}
\item Envelheça o vinho.
\item Opcionalmente redistribua os empregados (Enólogos para vinícolas e fazendeiros para vinhedos).
\item Produza vinho.
\end{itemize}

Se a qualidade, levando em consideração os modificadores como às condições climáticas for maior que 0 (zero) pegue um tile do estoque e armazene em seu depósito ou adega.

\subsection{Feira do vinho}
\label{sec:orgace1d09}

Nos anos 3, 5 e 6 temos a feira do vinho.

Os jogadores que não colocaram seu tile em um quiosque na feira devem fazê-lo agora em ordem de turno, respeitando as mesmas regras da coletiva de imprensa, porém a ordem de jogo não se altera e nem é possível
adicionar barris aos magnatas.

Então ocorre a feira:

\begin{itemize}
\item Pegue todos os seus tiles de especialista e secretamente escolha o número de especialistas permitidos pelo sei tile de feira.
\end{itemize}

Só é possível usar especialistas das características presentes em seu quiosque.

\begin{itemize}
\item Simultaneamente os jogadores revelam os especialistas selecionados.
\end{itemize}

Os especialistas que apresentarem uma seta no canto inferior esquerdo de seu tile, movem a respectiva característica do vinho para cima.

\begin{itemize}
\item Receba pontos de feira de acordo com o nível das características associadas aos seus especialistas apresentados.

\item Determine o ranking de acordo com o número acumulado de pontos de feira e atribua a pontuação de vitória.
\end{itemize}

Para 2 jogadores:

1a feira:

\begin{itemize}
\item 1o 9PV
\item 2o 3PV
\end{itemize}

2a feira

\begin{itemize}
\item 1o 12PV
\item 2o 4PV
\end{itemize}

3a feira

\begin{itemize}
\item 1o 15PV
\item 2o 5PV
\end{itemize}

Em caso de empate os pontos são divididos.

Final de feira:

\begin{itemize}
\item Descarte os especialistas usados.
\item Recupere seu tile de feira.
\item Ajuste a ordem de turno, o último colocado da feira é agora o primeiro.
\end{itemize}

\subsection{Final do jogo e pontuação}
\label{sec:orgafdd99d}

Depois da terceira feira do vinho começando com o jogador inicial (último colocado na feira), cada jogador usa o benefício de um magnata, descartando 1 vinho e movendo seu barril.
Continue até que todos os jogadores terminem suas possibilidades.

É permitido usar as habilidades bônus de seus especialistas restantes.

Finalmente receba pontos por:

\begin{itemize}
\item Seu balanço no banco.
\item Maioria nas colunas de exportação (empate os pontos são divididos).
\item Multiplicadores de final de jogo na área dos magnatas.
\end{itemize}

O jogador com mais PV é o vitorioso.

Saúde!!
\end{document}
