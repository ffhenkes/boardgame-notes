% Created 2024-07-24 qua 15:16
% Intended LaTeX compiler: pdflatex
\documentclass[11pt]{article}
\usepackage[utf8]{inputenc}
\usepackage[T1]{fontenc}
\usepackage{graphicx}
\usepackage{grffile}
\usepackage{longtable}
\usepackage{wrapfig}
\usepackage{rotating}
\usepackage[normalem]{ulem}
\usepackage{amsmath}
\usepackage{textcomp}
\usepackage{amssymb}
\usepackage{capt-of}
\usepackage{hyperref}
\author{Fabio Favero Henkes}
\date{\today}
\title{}
\hypersetup{
 pdfauthor={Fabio Favero Henkes},
 pdftitle={},
 pdfkeywords={},
 pdfsubject={},
 pdfcreator={Emacs 27.1 (Org mode 9.3)},
 pdflang={English}}
\begin{document}

\tableofcontents

\section{Arboretum}
\label{sec:orgdecfc23}

\subsection{Preparação}
\label{sec:orgd69acec}

\begin{itemize}
\item Em um jogo de dois jogadores 4 espécies devem ser removidas, ficando 6 espécies em jogo.
\item O jogador inicial é o último que irrigou uma planta!
\item O jogador à direita é o crupier
\item O crupier embaralha e distribui 7 cartas para cada jogador
\item O jogador sempre precisa ter em suas mão 7 cartas
\end{itemize}

\subsection{Jogo}
\label{sec:org9cc0829}

\begin{itemize}
\item Os jogadores se revezam em turnos no sentido horário
\item Em seu turno o jogador saca duas cartas do baralho ou das pilhas de descartes
\item O jogador deve baixar uma carta em seu Arboretum, ficar com outra em sua mão e descartar uma carta
\item Cada jogador possui uma pilha de descarte
\item O jogador pode sacar cartas de forma combinada entre todas as pilhas de descarte e/ou do baralho
\item Caso o jogador desejar sacar das pilhas de descarte, as cartas devem ser do topo
\item Ao baixar uma carta em seu Arboretum, isso deve ser feito na ortogonal e sempre de forma adjacente, diagonais não são permitidas
\item Uma vez baixada a carta não poderá mudar de posição
\item O jogo segue até o final do baralho
\end{itemize}

\subsection{Pontuação:}
\label{sec:orgbc977ae}

\subsubsection{Regras}
\label{sec:orge7e279a}

\begin{itemize}
\item Seguir sempre a partir de um espécie, deve iniciar e terminar com a mesma espécie
\item Pode haver cartas de espécies diferentes entre a carta inicial e final
\item Sequencias "limpas", da mesma espécie são vantajosas
\item O jogador escolhe o caminho que deseja traçar, de forma ortogonal, diagonais não são permitidas
\item Deve formar uma sequencia crescente
\item Não precisa ser uma sequencia perfeita, pode haver falhas entre os números 1-3-5-7-8 por ex
\item Uma mesma carta pode participar de mais de uma sequência (desde que parta de espécies diferentes, formando nova sequencia)
\end{itemize}

\subsubsection{Contabilização}
\label{sec:orgf2062e2}

\begin{itemize}
\item Cada carta da sequência concede 1 ponto
\item Caso termine em 8 ganha 2 pontos extras
\item Caso comece com 1 ganha 1 ponto extra
\item Caso a sequencia seja limpa ganha o dobro por carta (pontos extra não dobram)
\item Sequencias limpas (apenas da mesma espécie) precisam no mínimo de 4 cartas para pontuar dobrado, caso contrário pontuam normalmente.
\end{itemize}


\subsection{Quem pontua?}
\label{sec:orge06f11f}

\begin{itemize}
\item Para cada espécie, quem tiver na sua mão as cartas que batem com a sua sequencia formada da mesma espécie com a maior soma, pontua. Os demais jogadores zeram.
\item Caso haja empate, todos pontuam.
\item Caso nenhum dos jogadores possua cartas da espécie sendo avaliada, todos pontuam.
\item Caso o jogador tenha guardado em sua mão o 8 e outro jogador possua o 1, o 8 não vale nada.
\end{itemize}

\subsection{Quem vence?}
\label{sec:org57f0b41}

\begin{itemize}
\item Quem tiver o maior número de pontos ao final é o vencedor
\item Caso haja empate, quem tiver o maior número de espécies em seu arboretum vence
\end{itemize}
\end{document}
